\documentclass[12pt]{article}

\usepackage{sbc-template}
\usepackage{graphicx,url}
\usepackage[utf8]{inputenc}
\usepackage[brazil]{babel}
\usepackage[latin1]{inputenc}  
\usepackage[greek]{babel} 
\usepackage{alphabeta}

\sloppy

\title{Resumo Somatórios}

\author{696809 -- Marco Aurélio Silva de Souza Júnior}

\address{Faculdade de Ciência da Comutação -- PUC Minas\\ Av. Dom José Gaspar, 500, Coração Eucarístico - Belo Horizonte - MG, Brasil}

\begin{document} 

\maketitle

\section{Definição}
Somatórios são operadores matemáticos usados para simplificar uma sucessão de adições, finita ou infinita, com parcelas que respeitam algum padrão entre si. São denotados pela letra grega sigma maiúsculo ( $\sum_$ ).

Sua definição pode ser exposta como:

$\sum_{i=m}^{n}x_i = x_m + x_{m+1} + x_{m+2} + ... + x_{n-1} + x_n$\\
\\
Onde:

\emph{$x_{i}$}: variável indexada que representa cada termo do somatório;\par
\emph{m}: índice inicial;\par
\emph{n}: índice final; e\par
\emph{i}: índice do somatório, onde o primeiro valor de \textbf{i} é \textbf{m} e o último é \textbf{n}.

\section{Algumas propriedades}
\begin{enumerate}
    \item $\sum_{i=m}^{n}\alpha . x_i = \alpha \sum_{i=m}^{n}x_i$.
    \item $\sum_{i=m}^{n}(x_i \pm y_i) = \sum_{i=m}^{n}x_i \pm \sum_{i=m}^{n}x_i$.
    \item $\sum_{i=m}^{m}x_i = x_m$.
    \item $\sum_{i=m}^{n}x_i = \sum_{i=m}^{p}x_i + \sum_{i=p+1}^{n}x_i,  \forall m \leq p \leq n$.
    \item $\sum_{i=m}^{n} = \sum_{i=m+p}^{n+p}x_{i-p}$.
    \item $\sum_{i=m}^{n}(x_{i+1} - x_i) = x_{n+1} - x_m$.
    \item $\sum_{i=m}^{n} \sum_{j=k}^{l}x_i.y_i = \sum_{i=m}^{n}x_i\sum_{j=k}^{l}y_i$.
    \item $|\sum_{i=m}^{n}x_i| \leq \sum_{i=m}^{n}|x_i|$.
    \item $\sum_{n=0}^{t}x_{2n} + \sum_{n=0}^{t}x_{2n+1} = \sum_{n=0}^{2t+1}x_n$.
    \item $\sum_{n=0}^{t} \sum_{i=0}^{z-1} x_{z.n+i} = \sum_{n=0}^{z.t+z-1}x_n$.
    \item $(\sum_{k=0}^{n}a_k).(\sum_{k=0}^{n}b_k) = \sum_{k=0}^{2n}\sum_{i=0}^{k}a_i b_{k-i} - \sum_{k=0}^{n-1}(a_k \sum_{i=n+1}^{2n-k}b_i + b_k \sum_{i=n+1}^{2n-k}a_i)$.
\end{enumerate}

\end{document}
